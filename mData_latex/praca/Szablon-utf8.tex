%%====================================================================%%
%%                AKADEMIA GÓRNICZO-HUTNICZA W KRAKOWIE
%%                    WYDZIAŁ MATEMATYKI STOSOWANEJ
%%                          PRACA MAGISTERSKA
%%
%%       Autor : ----
%% Specjalność : ----
%%   Nr albumu : ----
%%    Promotor : ----
%%  Data (wer) : DD.MM.YYYY ...
%%
%%--------------------------------------------------------------------%%
%%
%%
%% ---------------
%%   PLIK GŁÓWNY
%% ---------------
%%
\documentclass[man,mfi|min|mpt|mok|mub|mza|miz|bsp]{mgrwms}

%%
%  Szablon dla kodowania UTF-8 przygotowany z pierwotnego szablonu
%  przy pomocy programu Gżegżółka 5.51.03 (www.gzegzolka.com)
%
%  Obowiązkowe opcje klasy:
%  - woman|man : wersja odpowiednio dla Pań i Panów,
%  - mfi|min|mpt|mok|mub|mza|miz|bsp : specjalności według skrótów:
%      * mfi :  Matematyka finansowa
%      * min :  Matematyka w informatyce
%      * mpt :  Matematyka w naukach technicznych i przyrodniczych
%      * mok :  Matematyka obliczeniowa i komputerowa
%      * mub :  Matematyka ubezpieczeniowa
%      * mza :  Matematyka w zarządzaniu
%      * miz :  Matematyka w informatyce i zarządzaniu
%      * bsp :  Bez specjalności
%  (opis pozostałych opcji znajduje się w dokumentacji 'UserGuide.pdf')
%%
%% ------- PAKIETY ------- %%
%%
\usepackage[utf8]{inputenc}      % kodowanie UTF-8
% \usepackage{amsmath}           % łatwiejszy skład matematyki
% \usepackage{amssymb}
% \usepackage{setspace}          % zmiana interlinii (np. 1,5 wiersza odstępu)
% \usepackage{makeidx}           % odkomentować, jeśli dołączony jest indeks
%
%  <... pozostale pakiety wedle uznania ...>
%
\usepackage{polski}         % pakiet polonizacyjny

\newcommand{\source}[1]{{Źródło: {#1}} }

\usepackage{color}
\usepackage{graphicx}
\usepackage{hyperref}
\usepackage{listings}

\lstset{
language=Python,
basicstyle=\small\sffamily,
numbers=left,
numberstyle=\tiny,
frame=tb,
%columns=fullflexible,
columns=fixed,
showstringspaces=false
}

%\titleclass{\subsubsubsection}{straight}[\subsection]

%\newcounter{subsubsubsection}[subsubsection]
%\renewcommand\thesubsubsubsection{\thesubsubsection.\arabic{subsubsubsection}}
%\renewcommand\theparagraph{\thesubsubsubsection.\arabic{paragraph}} % optional; useful if paragraphs are to be numbered

%\titleformat{\subsubsubsection}
%  {\normalfont\normalsize\bfseries}{\thesubsubsubsection}{1em}{}
%\titlespacing*{\subsubsubsection}
%{0pt}{3.25ex plus 1ex minus .2ex}{1.5ex plus .2ex}

\makeatletter
\renewcommand\paragraph{\@startsection{paragraph}{5}{\z@}%
  {3.25ex \@plus1ex \@minus.2ex}%
  {-1em}%
  {\normalfont\normalsize\bfseries}}
\renewcommand\subparagraph{\@startsection{subparagraph}{6}{\parindent}%
  {3.25ex \@plus1ex \@minus .2ex}%
  {-1em}%
  {\normalfont\normalsize\bfseries}}
\def\toclevel@subsubsubsection{4}
\def\toclevel@paragraph{5}
\def\toclevel@paragraph{6}
\def\l@subsubsubsection{\@dottedtocline{4}{7em}{4em}}
\def\l@paragraph{\@dottedtocline{5}{10em}{5em}}
\def\l@subparagraph{\@dottedtocline{6}{14em}{6em}}
\makeatother

\setcounter{secnumdepth}{4}
\setcounter{tocdepth}{4}


\graphicspath{ {/home/lukas/mgr/MData/mData_latex/schema/szablon/images/} }
% \usepackage[T1]{fontenc}  % lub z opcją 'QX', wymagane dla pakietu 'lmodern'
% \usepackage{lmodern}      % zalecane czcionki LModern
%%
%  Dodatkowe polecenia które trzeba umieścić w preambule:
%
% \makeindex  % jeśli dołączony jest indeks (nieobowiązkowy)
% \includeonly{...,...,...,...}  % dla warunkowej kompilacji
%%
%% BiBTeX (opcjonalnie)
% \bibliographystyle{ddabbrv}  % lub np. plabbrv, plalpha, plplain ...
% \nocite{*}
%%
%%
\begin{document}
%%
%%
%% ---------------
%%   NASZE MAKRA
%% ---------------
%%
%  Miejsce na nasze makra. Innym sposobem może być
%  umieszczenie ich w osobnym pliku i wczytanie poleceniem \input{----}
%%
%%
%% ---------------------------------
%%   METRYCZKA PRACY i SPIS TREŚCI
%% ---------------------------------
%%
\title{Analiza danych giełdowych przy pomocy narzędzi dostępnych w pakiecie scikit-learn}
\author{Łukasz Połoń}
\promotor{dr inż. Piotr Błaszyński}
\nralbumu{24942}
\slowakluczowe{Python, Scikit-learn, Giełda Papierów Wartościowych, Kivy, Matplotlib, Pandas, Analiza regresji}
\keywords{Python, Scikit-learn, Stock Market, Kivy, Matplotlib, Pandas, Regression analysis}
\maketitle
%%
%%
\tableofcontents
%%
%%
%% ----------------------
%%   STRESZCZENIA PRACY
%% ----------------------
%%
%% ------- POLSKIE ------- %%
%%
\begin{streszczenie}
%%
Przykładowe streszczenie i test polskich znaków: ąśćżźłóęą

%
%  <Treść streszczenia po polsku>
%
%%
\end{streszczenie}
%%
%% ------- ANGIELSKIE ------- %%
%%
\begin{abstract}
%%
%
%  <Treść streszczenia po angielsku>
%
%%
\end{abstract}
%%
%% ------- PODZIĘKOWANIA ------- %%
%%
%  <Jeśli zachodzi taka potrzeba, można dodać tutaj tekst podziękowania,
%   dedykacji itp.>
% \newpage
% \thispagestyle{empty}
%  <Treść podziękowania z odpowiednim formatowaniem>
%%
%%
%% ----------------------
%%   GŁÓWNY TEKST PRACY
%% ----------------------
%%
%% ------- WSTĘP ------- %%
%%
\begin{wstep}[Wprowadzenie]  % ew. np. \begin{wstep}[Wprowadzenie]
%%
Giełda Papierów Wartościowych jest to instytucja, która prowadzi działalność w zakresie organizacji obrotu papierami wartościowymi i instrumentami finansowymi\cite{gpw}.
W praktyce spełnia ona rolę pośrednika finansowego pomiędzy kupującym, a sprzedającym papiery wartościowe.
Dane generowane przez giełdę poddawane są ciągłym analizom, w szczególności w celu dostarczenia informacji potrzebnych do właściwego zarządzania kapitałem.\\
Istnieją dwie główne metody analizy danych giełdowych: analiza fundamentalna i analiza techniczna\cite{basics_of_tech_analysis}.
Pierwsza z nich polega na analizie faktycznej kondycji finansowej podmiotu, podczas gdy druga ma za zadanie prognozowanie przyszłych wartości wskaźników na podstawie zebranych danych.\\

Temat tej pracy podejmuje opisanie i przeprowadzenie wybranych metod analitycznych dostępnych w pakiecie scikit-learn. Pakiet ten jest biblioteką języka programowania Python umożliwiającą wysokopoziomowe przetwarzanie danych.
Udostępnia wiele algorytmów klasyfikacji, regresji oraz uczenia maszynowego, które mogą zostać wykorzystane do przeprowadzania obliczeń między innymi na potrzeby analizy technicznej, której to elementy zostaną tutaj przedstawione.

%%
\end{wstep}
%%
\input{chapter_1}

%\input{chapter_2}

%\input{chapter_3}

%\input{chapter_4}

%\input{chapter_5}

%% ------- ZAKOŃCZENIE ------- %%
%%
% \begin{zakonczenie}  % ew. np. \begin{zakonczenie}[Podsumowanie]
%%
%
%  <Treść zakończenia, podsumowania (jeśli jest taka potrzeba umieszczenia)>
%
%%
% \end{zakonczenie}
%%
%%
%% -----------------------
%%   ROZDZIAŁY DODATKOWE
%% -----------------------
%%
% \appendix
%%
%% ------- DODATEK A ------- %%
%%
% \chapter{----}
%%
%
%  <Treść dodatku A, może być umieszczona w innym pliku
%   i ładowana przez \input{----} lub \includeonly{...,...}>
%
%%
%%
%% ----------------
%%   BIBLIOGRAFIA
%% ----------------
%%
%  (Możemy użyć środowiska 'thebibliography' lub też bazy BiBTeX)
%%
%% BiBTeX (opcjonalnie)
% \bibliography{<pliki bib>}
%%
\begin{thebibliography}{88}  % ew. np. '8' jeśli liczba pozycji < 10
%% Przykładowe wrzuty

\bibitem{gpw} Tobiasz Maliński, 
\textit{Giełda Papierów Wartościowych Dla Bystrzaków}, Helion 2016.

\bibitem{basics_of_tech_analysis} Justin Kuepper,
\textit{Basics of Technical Analysis}, 19 Kwiecień 2017.
https://www.investopedia.com/university/technical/

\bibitem{stock_market} Investopedia,
\textit{Stock Market}, 20 Listopad 2017.
https://www.investopedia.com/terms/s/stockmarket.asp

\bibitem{ustawa_22_03_1991_papiery_wartosciowe}, Prawo o publicznym obrocie papierami wartościowymi i funduszach powierniczych
\textit{Ustawa z dnia 22 marca 1991r.}, Art 2.

\bibitem{akcje} Roman Ciepiela, Piotr Pytlik, Magda Wiernusz,
\textit{Encyklopedia Zarządzania}, 25 Październik 2016.
https://mfiles.pl/pl/index.php/Akcje

\bibitem{obligacje} Roman Ciepiela, Szymon Kułakowski, Sabina Blok,
\textit{Encyklopedia Zarządzania}, 13 Lipiec 2017
https://mfiles.pl/pl/index.php/Obligacje

%%
\end{thebibliography}
%%
%%
%% ----------------------------
%%   DODATKOWE ELEMENTY PRACY
%% ----------------------------
%%
%  Tutaj umieścimy dodatkowe (niebowiązkowe) cześci pracy:
%  - Spis skrótów, symboli i oznaczeń (jako \chapter*{...})
%  - Spis ilustracji (\listoffigures)
%  - Spis tablic (\listoftables)
%  - Skorowidz (nieobowiązkowy, umieszczenie ułatwia lekturę pracy)
%
% \chapter*{Spis symboli i oznaczeń}  % ew. podobny tytuł
%  
%   <Tabela ze spisem symboli, oznaczeń itp.>
%
\listoffigures
%
\listoftables
%
%\printindex  % jeżeli do pracy dołączony jest indeks
%%
%%
\end{document}
%%
%% [Wersja dla: 2011/09/06 v1.06]
%%====================================================================%%
