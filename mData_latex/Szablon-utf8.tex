%%====================================================================%%
%%                AKADEMIA GÓRNICZO-HUTNICZA W KRAKOWIE
%%                    WYDZIAŁ MATEMATYKI STOSOWANEJ
%%                          PRACA MAGISTERSKA
%%
%%       Autor : ----
%% Specjalność : ----
%%   Nr albumu : ----
%%    Promotor : ----
%%  Data (wer) : DD.MM.YYYY ...
%%
%%--------------------------------------------------------------------%%
%%
%%
%% ---------------
%%   PLIK GŁÓWNY
%% ---------------
%%
%\documentclass[man,mfi|min|mpt|mok|mub|mza|miz|bsp]{mgrwms}
\documentclass[man,mfi]{mgrwms}
%%
%  Szablon dla kodowania UTF-8 przygotowany z pierwotnego szablonu
%  przy pomocy programu Gżegżółka 5.51.03 (www.gzegzolka.com)
%
%  Obowiązkowe opcje klasy:
%  - woman|man : wersja odpowiednio dla Pań i Panów,
%  - mfi|min|mpt|mok|mub|mza|miz|bsp : specjalności według skrótów:
%      * mfi :  Matematyka finansowa
%      * min :  Matematyka w informatyce
%      * mpt :  Matematyka w naukach technicznych i przyrodniczych
%      * mok :  Matematyka obliczeniowa i komputerowa
%      * mub :  Matematyka ubezpieczeniowa
%      * mza :  Matematyka w zarządzaniu
%      * miz :  Matematyka w informatyce i zarządzaniu
%      * bsp :  Bez specjalności
%  (opis pozostałych opcji znajduje się w dokumentacji 'UserGuide.pdf')
%%
%% ------- PAKIETY ------- %%
%%
\usepackage[utf8]{inputenc}      % kodowanie UTF-8
% \usepackage{amsmath}           % łatwiejszy skład matematyki
% \usepackage{amssymb}
% \usepackage{setspace}          % zmiana interlinii (np. 1,5 wiersza odstępu)
% \usepackage{makeidx}           % odkomentować, jeśli dołączony jest indeks
%
%  <... pozostale pakiety wedle uznania ...>
%
\usepackage{polski}         % pakiet polonizacyjny
% \usepackage[T1]{fontenc}  % lub z opcją 'QX', wymagane dla pakietu 'lmodern'
% \usepackage{lmodern}      % zalecane czcionki LModern
%%
%  Dodatkowe polecenia które trzeba umieścić w preambule:
%
% \makeindex  % jeśli dołączony jest indeks (nieobowiązkowy)
% \includeonly{...,...,...,...}  % dla warunkowej kompilacji
%%
%% BiBTeX (opcjonalnie)
% \bibliographystyle{ddabbrv}  % lub np. plabbrv, plalpha, plplain ...
% \nocite{*}
%%
%%
\begin{document}
%%
%%
%% ---------------
%%   NASZE MAKRA
%% ---------------
%%
%  Miejsce na nasze makra. Innym sposobem może być
%  umieszczenie ich w osobnym pliku i wczytanie poleceniem \input{----}
%%
%%
%% ---------------------------------
%%   METRYCZKA PRACY i SPIS TREŚCI
%% ---------------------------------
%%
\title{Analiza i porównanie wybranych właściwości języka Python na poziomie kodu pośredniego}
\author{Maciej Tomczuk}
\promotor{dr inż. Piotr Błaszyński}
\nralbumu{24022}
\slowakluczowe{Python, kod pośredni, disassembler}
\keywords{Python, bytecode, disassembler}
%%
\maketitle
%%
%%
\tableofcontents
%%
%%
%% ----------------------
%%   STRESZCZENIA PRACY
%% ----------------------
%%
%% ------- POLSKIE ------- %%
%%
\begin{streszczenie}
%%
Przykładowe streszczenie i test polskich znaków: ąśćżźłóęą
%
%  <Treść streszczenia po polsku>
%
%%
\end{streszczenie}
%%
%% ------- ANGIELSKIE ------- %%
%%
\begin{abstract}
%%
%
%  <Treść streszczenia po angielsku>
%
%%
\end{abstract}
%%
%% ------- PODZIĘKOWANIA ------- %%
%%
%  <Jeśli zachodzi taka potrzeba, można dodać tutaj tekst podziękowania,
%   dedykacji itp.>
% \newpage
% \thispagestyle{empty}
%  <Treść podziękowania z odpowiednim formatowaniem>
%%
%%
%% ----------------------
%%   GŁÓWNY TEKST PRACY
%% ----------------------
%%
%% ------- WSTĘP ------- %%
%%
\begin{wstep}  % ew. np. \begin{wstep}[Wprowadzenie]
%%
%
%  <Treść wstępu>
%
%%
\end{wstep}
%%
\input{chapter_1}

\input{chapter_2}

%% ------- ZAKOŃCZENIE ------- %%
%%
% \begin{zakonczenie}  % ew. np. \begin{zakonczenie}[Podsumowanie]
%%
%
%  <Treść zakończenia, podsumowania (jeśli jest taka potrzeba umieszczenia)>
%
%%
% \end{zakonczenie}
%%
%%
%% -----------------------
%%   ROZDZIAŁY DODATKOWE
%% -----------------------
%%
% \appendix
%%
%% ------- DODATEK A ------- %%
%%
% \chapter{----}
%%
%
%  <Treść dodatku A, może być umieszczona w innym pliku
%   i ładowana przez \input{----} lub \includeonly{...,...}>
%
%%
%%
%% ----------------
%%   BIBLIOGRAFIA
%% ----------------
%%
%  (Możemy użyć środowiska 'thebibliography' lub też bazy BiBTeX)
%%
%% BiBTeX (opcjonalnie)
% \bibliography{<pliki bib>}
%%
\begin{thebibliography}{88}  % ew. np. '8' jeśli liczba pozycji < 10
%%
%
% Treść bibliografii, wpisy typu:
% \bibitem{<Klucz do cytowań>} ...
% \bibitem{<Klucz do cytowań>} ...
%
%%
\end{thebibliography}
%%
%%
%% ----------------------------
%%   DODATKOWE ELEMENTY PRACY
%% ----------------------------
%%
%  Tutaj umieścimy dodatkowe (niebowiązkowe) cześci pracy:
%  - Spis skrótów, symboli i oznaczeń (jako \chapter*{...})
%  - Spis ilustracji (\listoffigures)
%  - Spis tablic (\listoftables)
%  - Skorowidz (nieobowiązkowy, umieszczenie ułatwia lekturę pracy)
%
% \chapter*{Spis symboli i oznaczeń}  % ew. podobny tytuł
%  
%   <Tabela ze spisem symboli, oznaczeń itp.>
%
% \listoffigures
%
% \listoftables
%
% \printindex  % jeżeli do pracy dołączony jest indeks
%%
%%
\end{document}
%%
%% [Wersja dla: 2011/09/06 v1.06]
%%====================================================================%%
